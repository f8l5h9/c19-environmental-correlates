\documentclass[]{elsarticle} %review=doublespace preprint=single 5p=2 column
%%% Begin My package additions %%%%%%%%%%%%%%%%%%%
\usepackage[hyphens]{url}

  \journal{Geographical Analysis} % Sets Journal name


\usepackage{lineno} % add
\providecommand{\tightlist}{%
  \setlength{\itemsep}{0pt}\setlength{\parskip}{0pt}}

\usepackage{graphicx}
\usepackage{booktabs} % book-quality tables
%%%%%%%%%%%%%%%% end my additions to header

\usepackage[T1]{fontenc}
\usepackage{lmodern}
\usepackage{amssymb,amsmath}
\usepackage{ifxetex,ifluatex}
\usepackage{fixltx2e} % provides \textsubscript
% use upquote if available, for straight quotes in verbatim environments
\IfFileExists{upquote.sty}{\usepackage{upquote}}{}
\ifnum 0\ifxetex 1\fi\ifluatex 1\fi=0 % if pdftex
  \usepackage[utf8]{inputenc}
\else % if luatex or xelatex
  \usepackage{fontspec}
  \ifxetex
    \usepackage{xltxtra,xunicode}
  \fi
  \defaultfontfeatures{Mapping=tex-text,Scale=MatchLowercase}
  \newcommand{\euro}{€}
\fi
% use microtype if available
\IfFileExists{microtype.sty}{\usepackage{microtype}}{}
\bibliographystyle{elsarticle-harv}
\ifxetex
  \usepackage[setpagesize=false, % page size defined by xetex
              unicode=false, % unicode breaks when used with xetex
              xetex]{hyperref}
\else
  \usepackage[unicode=true]{hyperref}
\fi
\hypersetup{breaklinks=true,
            bookmarks=true,
            pdfauthor={},
            pdftitle={A spatial analysis of the environmental correlates of COVID-19 incidence in the provinces in Spain},
            colorlinks=false,
            urlcolor=blue,
            linkcolor=magenta,
            pdfborder={0 0 0}}
\urlstyle{same}  % don't use monospace font for urls

\setcounter{secnumdepth}{0}
% Pandoc toggle for numbering sections (defaults to be off)
\setcounter{secnumdepth}{0}


% Pandoc header



\begin{document}
\begin{frontmatter}

  \title{A spatial analysis of the environmental correlates of COVID-19 incidence
in the provinces in Spain}
    \author[University]{Author A\corref{1}}
   \ead{authora@institution.edu} 
    \author[School]{Autor B}
   \ead{bob@example.com} 
      \address[University]{Department, Street, City, State, Zip}
    \address[School]{Department, Street, City, State, Zip}
      \cortext[1]{Corresponding Author}
    \cortext[2]{Equal contribution}
  
  \begin{abstract}
  Spreading with astonishing speed, the novel SARS-CoV2 has swept the
  globe, causing enormous stress to health systems and prompting social
  distance guidelines and mandates to arrest its progress. While there is
  encouraging evidence that early public health interventions have slowed
  the spread of the virus, this has come at a high cost as the global
  economy is brought to its knees. How and when to ease restrictions to
  movement hinges in part on the question whether SARS-CoV2 will display
  seasonality associated with variations in temperature and humidity. In
  this research, we address this question by means of a spatial analysis
  of the incidence of COVID-19 in the provinces in Spain. Use of a spatial
  SUR approach allows us to model the incidence of reported cases of the
  disease per 100,000 population, as a function of temperature and
  humidity, while controlling for GDP per capita, population density,
  percentage of older adults in the population, and presence of mass
  transit systems. Our results indicate that incidence of the disease is
  lower at higher temperatures. The evidence with respect to humidity is
  more mixed, with coefficients for this variable that are significant in
  only some equations. Our control variables also yield interesting
  insights, as population density and percentage of older adults display
  negative associations with incidence of COVID-19, whereas the presence
  of mass transit systems in the province is associated with a greater
  incidence of the disease.
  \end{abstract}
  
 \end{frontmatter}

\hypertarget{introduction}{%
\section{Introduction}\label{introduction}}

From a small outbreak linked to a live animal market at the end of 2019
to a global pandemic in a matter of weeks, the SARS-CoV2 virus has
threatened to overrun health systems the world over. In efforts to
contain the spread, numerous governments in many nations and regions
have either recommended or mandated social distancing measures, and as
of this writing, millions of people in five continents shelter in place.
There are encouraging signs that these measures have arrested the spread
of the virus where they have been implemented, and have thus helped to
keep a bad situation from becoming even worse (e.g., 2020). However,
this has come at a high cost, and the consequences for all spheres of
the economy, social, and cultural life have been dire (e.g., Fernandes,
2020; Luo and Tsang, 2020). As a result, there is a sense of urgency to
anticipate the progression of the pandemic, in order to plan for
progressive lifting of restrictions to movement and social contact
(e.g., Kissler et al., 2020). Needless to say, this has become a
delicate, and politically charged, balancing act between public health
and the economy (Gong et al., 2020).

A salient question in the debate on how and when to ease social
distancing measures is whether the virus will display seasonal
variations. Earlier, diverse studies have shown the effect of
temperature and humidity on the incidence of influenza (e.g., Mäkainen
et al., 2009; Jaakkola et al., 2014; Kudo et al., 2019). Jaakkola et
al.~(2014), for example, found that a decrease of temperature and
absolute humidity precedes the onset of symptoms of influenza A and B
viruses by 3 days in places where the temperature is low. After the
2002-2004 outbreak of SARS, researchers also began to investigate the
relationship between these factors and SARS-CoV. In this way, Casanova
et al.~(2010) used two surrogates, namely the gastroenteritis (TGEV) and
mouse hepatitis viruses (MHV), to find that virus inactivation was more
rapid at temperatures of 20C than 4C, and at 40C than 20C; in terms of
humidity, these researchers reported that survival of the virus was
lower at moderate relative humidity levels. In a similar vein, but
working directly with SARS-CoV, Chan et al.~(2011) found that viability
of the virus was lost at temperatures higher than 38C and relative
humidity superior to 95\%.

While existing research on similar pathogens suggests that SARS-CoV is
more stable and potentially easier to transmit in conditions of low
temperature and low humidity, it is far from certain that this will also
be the case with the novel SARS-CoV2. If such is the case, there is the
possibility of easing restrictions to social contact as the weather
warms; however, weeks or possibly months of costly measures could become
undone if not, and the restrictions are lifted prematurely. Not
surprisingly, this issue has triggered a lively debate.

In this paper, we report results from a spatial analysis of incidence of
COVID-19 in fifty provinces in Spain. Spain is one of the countries
hardest hit by the virus, and enacted lockdown measures on March 16,
2020, in response to a rapidly growing outbreak of COVID-19. We combine
data on reported cases of the disease with metereological information,
to create a spatio-temporal dataset covering a period of 30 days. We
then join this dataset with provincial-level economic and demographic
information to act as controls to our key environmental variables. These
data are analyzed using a spatial SUR approach, which allows us to
account for residual spatial autocorrelation. The results provide
persuasive evidence of the effect of temperature on the incidence of
COVID-19, as \textbf{NOTE ABOUT THE MAGNITUDE OF THE EFFECT}. The
evidence concerning humidity is more mixed: while the direction of the
effect is negative, as anticipated, the coefficients for this variable
are only significant in some of equations in the system. Our control
variables also provide some intriguing insights. The results of this
analysis provide support to the hypothesis of seasonality of the novel
SARS-CoV2, and should be of interest to public health officials and
policy makers wrestling with the question of the trajectory of the
pandemic.

\hypertarget{context-and-data}{%
\section{Context and Data}\label{context-and-data}}

\hypertarget{covid-19-in-spain}{%
\subsection{Covid-19 in Spain}\label{covid-19-in-spain}}

Background information go here.

\hypertarget{sources-of-data-and-data-preparation}{%
\subsection{Sources of data and data
preparation}\label{sources-of-data-and-data-preparation}}

Explain the sources of data and data preprocessing.

\hypertarget{methods-spatial-sur}{%
\section{Methods: Spatial SUR}\label{methods-spatial-sur}}

Do not forget to plug-in a reference to the package for spatial sur.

\hypertarget{analysis-and-results}{%
\section{Analysis and Results}\label{analysis-and-results}}

The literature about COVID-19 suggested that population density is the
one of the most important proliferate cause of these viscous, however
this ill spread with different intensity at big cities of the world.
Controlling for socioeconomic characteristics the objective of this
paper is observe the effect of clime on COVID-19 proliferation.

\hypertarget{discussion}{%
\section{Discussion}\label{discussion}}

Possibly do some simulations with the model.

\hypertarget{concluding-remarks}{%
\section{Concluding Remarks}\label{concluding-remarks}}

More words go here.

\hypertarget{references}{%
\section*{References}\label{references}}
\addcontentsline{toc}{section}{References}

\hypertarget{refs}{}
\leavevmode\hypertarget{ref-Casanova2010effects}{}%
Casanova, L.M., Jeon, S., Rutala, W.A., Weber, D.J., Sobsey, M.D., 2010.
Effects of air temperature and relative humidity on coronavirus survival
on surfaces. Appl. Environ. Microbiol. 76, 2712--2717.

\leavevmode\hypertarget{ref-Chan2011effects}{}%
Chan, K., Peiris, J., Lam, S., Poon, L., Yuen, K., Seto, W., 2011. The
effects of temperature and relative humidity on the viability of the
sars coronavirus. Advances in virology 2011.

\leavevmode\hypertarget{ref-Fernandes2020economic}{}%
Fernandes, N., 2020. Economic effects of coronavirus outbreak (covid-19)
on the world economy. Available at SSRN 3557504.

\leavevmode\hypertarget{ref-Gong2020balance}{}%
Gong, B., Zhang, S., Yuan, L., Chen, K.Z., 2020. A balance act:
Minimizing economic loss while controlling novel coronavirus pneumonia.
Journal of Chinese Governance 1--20.

\leavevmode\hypertarget{ref-Jaakkola2014decline}{}%
Jaakkola, K., Saukkoriipi, A., Jokelainen, J., Juvonen, R., Kauppila,
J., Vainio, O., Ziegler, T., Rönkkö, E., Jaakkola, J.J., Ikäheimo, T.M.,
2014. Decline in temperature and humidity increases the occurrence of
influenza in cold climate. Environmental Health 13, 22.

\leavevmode\hypertarget{ref-Kissler2020projecting}{}%
Kissler, S.M., Tedijanto, C., Goldstein, E., Grad, Y.H., Lipsitch, M.,
2020. Projecting the transmission dynamics of sars-cov-2 through the
postpandemic period. Science eabb5793.
doi:\href{https://doi.org/10.1126/science.abb5793}{10.1126/science.abb5793}

\leavevmode\hypertarget{ref-Kudo2019low}{}%
Kudo, E., Song, E., Yockey, L.J., Rakib, T., Wong, P.W., Homer, R.J.,
Iwasaki, A., 2019. Low ambient humidity impairs barrier function and
innate resistance against influenza infection. Proceedings of the
National Academy of Sciences 116, 10905--10910.

\leavevmode\hypertarget{ref-Lancastle2020impact}{}%
Lancastle, N.M., 2020. Is the impact of social distancing on coronavirus
growth rates effective across different settings? A non-parametric and
local regression approach to test and compare the growth rate. medRxiv.

\leavevmode\hypertarget{ref-Luo2020how}{}%
Luo, S., Tsang, K.P., 2020. How much of china and world gdp has the
coronavirus reduced? Available at SSRN 3543760.

\leavevmode\hypertarget{ref-Makinen2009cold}{}%
Mäkainen, T.M., Juvonen, R., Jokelainen, J., Harju, T.H., Peitso, A.,
Bloigu, A., Silvennoinen-Kassinen, S., Leinonen, M., Hassi, J., 2009.
Cold temperature and low humidity are associated with increased
occurrence of respiratory tract infections. Respiratory medicine 103,
456--462.


\end{document}


